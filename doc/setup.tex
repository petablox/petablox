\xname{setup}
\chapter{Setting up a Java Program for Analysis}
\label{chap:setup}

This chapter describes how to setup a Java program for analysis using Petablox.
Suppose the program has the following directory structure:

\begin{framed}
\begin{verbatim}
example/
    src/
        foo/
            Main.java
            ...
    classes/
        foo/
            Main.class
            ...
    lib/
        src/
            taz/
                ...
        jar/
            taz.jar
    petablox.properties
\end{verbatim}
\end{framed}

The above structure is typical: the program's Java source
files are under {\tt src/}, its class files are under {\tt classes/},
and the source and jar files of the libraries used by the program are
under \code{lib/src/} and \code{lib/jar/}, respectively.  The
purpose of the \code{petablox.properties} file is explained below.

The only way to specify inputs to Petablox, including the program
to be analyzed, is via system properties.
Section \ref{sec:properties-setting} describes various ways by which
properties can be passed to Petablox.  Here, we describe the
simplest approach, in which all properties of the program to be analyzed
that might be needed by Petablox are defined in a file named \code{petablox.properties} 
that is located in the top-level directory of the program (directory \code{example/} above).
Then, Petablox can be applied to the program by running the following command:

\begin{framed}
\begin{verbatim}
ant -Dpetablox.work.dir=<WORK_DIR> run
\end{verbatim}
\end{framed}

This command instructs Petablox to run in the directory denoted by \code{<WORK_DIR>}, where it searches for a file
named \code{petablox.properties} and
loads all properties defined in that file, if it exists.
Thus, for the above program, \code{<WORK_DIR>} must be the absolute or relative path of the
\code{example/} directory.  A sample \code{petablox.properties} file for the above program is as follows:

\begin{framed}
\begin{verbatim}
petablox.main.class=foo.Main
petablox.class.path=classes:lib/jar/taz.jar
petablox.src.path=src:lib/src
petablox.run.ids=0,1
petablox.args.0="-thread 1 -n 10"
petablox.args.1="-thread 2 -n 50"
\end{verbatim}
\end{framed}

Each relative file/directory name in the value of any property
defined in this file (e.g., the \code{lib/src} directory name in the value of
property \code{petablox.src.path} above) is treated relative to the directory
specified by property \code{petablox.work.dir}, whose default value
is the current directory.
Section \ref{sec:program-props} presents all program properties that are
recognized by Petablox.  Here, we only describe those that are most commonly
used, namely, those defined in the above sample properties file:

\begin{itemize}
\item
\code{petablox.main.class} specifies the fully-qualified name of the main
class of the program.
\item
\code{petablox.class.path} specifies the application classpath
of the program (the JDK standard library classpath is implicitly
included).
\item
\code{petablox.src.path} specifies the Java source path of the program.
All analyses in Petablox operate on Java bytecode.  The only use
of this property is to HTMLize the Java source files of the program so
that the results of analyses can be reported at the Java
source code level.
\item
\code{petablox.run.ids} specifies a list of IDs to identify runs of the
program.  It is used by dynamic analyses to determine how many
times the program must be run.  An additional use of this property is
to allow specifying the command-line arguments to use in the run
having ID {\tt <id>} via property \code{petablox.args.<id>}, as
illustrated by properties \code{petablox.args.0} and \code{petablox.args.1}
above.
\end{itemize}

The above command does not do much beyond making Petablox load the above
properties file.  For Petablox to do something interesting,
additional properties must be set that specify the function(s)
Petablox must perform.  All functions are summarized in Section \ref{sec:func-props}.
The most common function is to run one or more analyses on the input program;
it is described in Chapter \ref{chap:running}.

